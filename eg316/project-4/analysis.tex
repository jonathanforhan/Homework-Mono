% Jonathan Forhan
% EG-316 Report -- Project 4

\documentclass[12pt]{article}

% title section
\title{EG-316 Project 4}
\author{Jonathan Forhan}
\date{ }

% packages
\usepackage[american]{circuitikz}
\usepackage[margin=0.6in]{geometry}
\usepackage{amsmath}
\usepackage{caption}
\usepackage{graphicx}
\usepackage{multicol}
\usepackage{pgfplots}
\usepackage{siunitx}
\usepackage{tocloft}

% mu
\sisetup{math-micro=\text{μ},text-micro=μ}

% add dots to table of contents
\renewcommand{\cftsecleader}{\cftdotfill{\cftdotsep}}

% no sci notation
% \pgfplotsset{scaled y ticks=false}
% \pgfplotsset{scaled x ticks=false}
\pgfplotsset{compat=1.18}

% unbolden components
\ctikzset{bipoles/thickness=1}

%% Macros
\def\k{\mathrm{k}}
\def\V{\mathrm{V}}
\def\A{\mathrm{A}}
\def\m{\mathrm{m}}
\def\u{\textrm{\textmu}}

%% start document
\begin{document}

% title page
\maketitle
\tableofcontents
\thispagestyle{empty}
\clearpage

\section{Introduction}

This project will use various circuit analysis techniques learned throughout the semester to determine the open-circuit voltage, short-circuit current
and Th\'evenin resistance of following circuit:

\begin{figure}[ht]
	\begin{center}
		\begin{circuitikz}
			% NW
			\draw
			(0,3) to[R, l_=$33\k\Omega$]
			(0,6) to[R, l=$6.8\k\Omega$]
			(4,6) to[V, l=$6\V$]
			(4,3) to[R, l_=$15\k\Omega$]
			(0,3);

			% NE
			\draw
			(4,6) to[R, l=$2.2\k\Omega$]
			(8,6) to[R, l=$1\k\Omega$]
			(8,3) --
			(4,3);

			\draw
			(8,6) to [short, -o]
			(9,6) ++ (16pt,0) node[left] {a};

			% SW
			\draw
			(0,3) to[V, l=$6\V$]
			(0,0) --
			(4,0) to[R, l_=$47\k\Omega$]
			(4,3);

			\draw
			(2,0) to (2,-0.25) node[tlground] {};

			% SE
			\draw
			(4,0) to [R, l=$560\Omega$]
			(8,0);

			\draw
			(8,0) to [short, -o]
			(9,0) ++ (16pt,0) node[left] {b};
		\end{circuitikz}
		\caption{Circuit with no load}
	\end{center}
\end{figure}

\clearpage

\section{Open-circuit voltage}

\subsection{Nodal analysis}

\begin{figure}[ht]
	\begin{center}
		\begin{circuitikz}
			% NW
			\draw
			(0,3) to[R, l_=$33\k\Omega$]
			(0,6) to[R, l=$6.8\k\Omega$]
			(4,6) to[V, l=$6\V$]
			(4,3) to[R, l_=$15\k\Omega$]
			(0,3);

			% NE
			\draw
			(4,6) to[R, l=$2.2\k\Omega$]
			(8,6) to[R, l=$1\k\Omega$]
			(8,3) --
			(4,3);

			\draw
			(8,6) to [short, -o]
			(9,6) ++ (16pt,0) node[left] {a};

			% SW
			\draw
			(0,3) to[V, l=$6\V$]
			(0,0) --
			(4,0) to[R, l_=$47\k\Omega$]
			(4,3);

			\draw
			(2,0) to (2,-0.25) node[tlground] {};

			% SE
			\draw
			(4,0) to [R, l=$560\Omega$]
			(8,0);

			\draw
			(8,0) to [short, -o]
			(9,0) ++ (16pt,0) node[left] {b};

			% nodal
			\draw (4,6) to [short, -*] (4,6);
			\draw (4,6.5) ++ (16pt,0) node[left] {$V_1$};
			\draw (4,3) to [short, -*] (4,3);
			\draw (4,3.5) ++ (16pt,0) node {$V_2$};
			\draw (0,3) to [short, -*] (0,3);
			\draw (-1,3) ++ (16pt,0) node[left] {$V_3$};
		\end{circuitikz}
		\caption{Open-circuit with nodes added}
	\end{center}
\end{figure}

\begin{enumerate}
	\item The number of nodal equations required equals: \\ (essential nodes - ground node - voltage source) $= 4-1-2=1$
	\item Determine our node locations, marked on the diagram $V_{(1-3)}$ we can easily deduced $V_3=6\V$, and $V_1=V_2+6\V$
	\item $V_1$ and $V_2$ make a supernode, knowing that we can now write our 1 equation for the current entering the supernode with $V_2$ as our unknown.
	\item $\displaystyle{
			      \frac{(V_2+6\V)-6\V}{6.8\k\Omega+33\k\Omega}
			      +\frac{(V_2+6\V)-V_2}{3.2\k\Omega}
			      +\frac{V_2-(V_2+6\V)}{3.2\k\Omega}
			      +\frac{V_2}{47\k\Omega}
			      +\frac{V_2-6\V}{15\k\Omega}}=0$
	\item $\displaystyle{\frac{V_2}{39.8\k\Omega}+\frac{V_2}{47\k\Omega}+\frac{V_2}{15\k\Omega}-\frac{6\V}{15\k\Omega}=0}$
	\item $\displaystyle{V_2\left(\frac{1}{39.8\k\Omega}+\frac{1}{47\k\Omega}+\frac{1}{15\k\Omega}\right)=\frac{6\V}{15\k\Omega}}$
	\item $\displaystyle{V_2=3.538\V}$, $\displaystyle{V_1=9.538\V}$
	\item Now that we know $V_2$, we can solve for the node $a, \displaystyle{V_a=\frac{6\V\cdot 1\k\Omega}{1\k\Omega+2.2\k\Omega}+V_2=5.413\V}$
	\item $V_b$ is 0V because there is no current through the $540\Omega$ resistor, thus $V_{ab}=5.41\V$
\end{enumerate}

\clearpage

\subsection{Mesh analysis}

\begin{figure}[ht]
	\begin{center}
		\begin{circuitikz}
			% NW
			\draw
			(0,3) to[R, l_=$33\k\Omega$]
			(0,6) to[R, l=$6.8\k\Omega$]
			(4,6) to[V, l=$6\V$]
			(4,3) to[R, l_=$15\k\Omega$]
			(0,3);

			% NE
			\draw
			(4,6) to[R, l=$2.2\k\Omega$]
			(8,6) to[R, l=$1\k\Omega$]
			(8,3) --
			(4,3);

			\draw
			(8,6) to[short, -o]
			(9,6) ++(16pt,0) node[left] {a};

			% SW
			\draw
			(0,3) to[V, l=$6V$]
			(0,0) --
			(4,0) to[R, l_=$47\k\Omega$]
			(4,3);

			\draw
			(2,0) to (2,-0.25) node[tlground] {};

			% SE
			\draw
			(4,0) to [R, l=$560\Omega$]
			(8,0);

			\draw
			(8,0) to [short, -o]
			(9,0) ++ (16pt,0) node[left] {b};

			% mesh
			\draw[->] (1.6,5) arc(120:-120:6mm) node[midway, left, font=\footnotesize] {$i_1(t)$};
			\draw[->] (1.6,2) arc(120:-120:6mm) node[midway, left, font=\footnotesize] {$i_2(t)$};
			\draw[->] (5.6,5) arc(120:-120:6mm) node[midway, left, font=\footnotesize] {$i_3(t)$};
		\end{circuitikz}
		\caption{Open-circuit with mesh currents added}
	\end{center}
\end{figure}

\begin{enumerate}
	\item Determine our mesh currents labeled $i_{1-3}$
	\item \begin{align*}
		      0 & =33\k\Omega(i_1)+6.8\k\Omega(i_1)+6\V+15\k\Omega(i_1-i_2) \\
		      0 & =-6\V+15\k\Omega(i_2-i_1)+47\k\Omega(i_2)                 \\
		      0 & =-6\V+2.2\k\Omega(i_3)+1\k\Omega(i_3)
	      \end{align*}
	\item \begin{align*}
		      -6\V & = 54.8\k\Omega(i_1)-15\k\Omega(i_2) \\
		      6\V  & = -15\k\Omega(i_1)+62\k\Omega(i_2)  \\
		      6\V  & = 3.2\k\Omega(i_3)
	      \end{align*}
	\item \[
		      \begin{bmatrix}
			      54.8\k\Omega & -15\k\Omega & 0           \\
			      -15\k\Omega  & 62\k\Omega  & 0           \\
			      0            & 0           & 3.2\k\Omega
		      \end{bmatrix}
		      \begin{bmatrix}
			      i_1 \\
			      i_2 \\
			      i_3
		      \end{bmatrix}
		      =
		      \begin{bmatrix}
			      -6\V \\
			      6\V  \\
			      6\V
		      \end{bmatrix}
	      \]
	\item $i_1=-88.9\u\A$, $i_2=75.3\u\A$, $i_3=1.88\m\A$
	\item There is no current through the $560\Omega$ resistor so we know $V_{ab}=1\k\Omega(i_3)+47\k\Omega(i_2)$
	\item $V_{ab}=1\k\Omega(1.88\m\A)+47\k\Omega(75.3\u\A)=5.41\V$
\end{enumerate}

\clearpage

\subsection{Superposition}

\begin{multicols}{2}
	\begin{center}
		\begin{circuitikz}
			% NW
			\draw
			(0,3) to[R, l_=$33\k\Omega$]
			(0,6) to[R, l=$6.8\k\Omega$]
			(3,6) --
			(3,3) to[R, l_=$15\k\Omega$]
			(0,3);
			% NE
			\draw
			(3,6) to[R, l=$2.2\k\Omega$]
			(6,6) --
			(6,3) to[R, l_=$1\k\Omega$]
			(3,3);
			\draw
			(6,3) to [short, -o]
			(6,2) ++ (-16pt,0) node[right] {a};
			% SW
			\draw
			(0,3) to[V, l=$6\V$]
			(0,0) --
			(3,0) to[R, l_=$47\k\Omega$]
			(3,3);
			\draw
			(2,0) to (2,-0.25) node[tlground] {};
			\draw
			(3,0) to [short, -o]
			(4,0) ++ (16pt,0) node[left] {b};
		\end{circuitikz}
		\captionof{figure}{Open-circuit with $V_\alpha$ isolated}
	\end{center}
	\begin{center}
		\begin{circuitikz}
			% NW
			\draw
			(0,3) to[R, l_=$33\k\Omega$]
			(0,6) to[R, l=$6.8\k\Omega$]
			(3,6) to[V, l=$6\V$]
			(3,3) to[R, l_=$15\k\Omega$]
			(0,3);
			% NE
			\draw
			(3,6) to[R, l=$2.2\k\Omega$]
			(6,6) --
			(6,3) to[R, l_=$1\k\Omega$]
			(3,3);
			\draw
			(6,3) to [short, -o]
			(6,2) ++ (-16pt,0) node[right] {a};
			% SW
			\draw
			(0,3) --
			(0,0) --
			(3,0) to[R, l_=$47\k\Omega$]
			(3,3);
			\draw
			(2,0) to (2,-0.25) node[tlground] {};
			\draw
			(3,0) to [short, -o]
			(4,0) ++ (16pt,0) node[left] {b};
		\end{circuitikz}
		\captionof{figure}{Open-circuit with $V_\beta$ isolated}
	\end{center}
\end{multicols}

\begin{multicols}{2}
	\begin{enumerate}
		\item The top-right loop has no current through it, so it can be elided.
		\item The top-left loop can now be simplified: $R_{eq}=(33\k\Omega+6.8\k\Omega)||15\k\Omega=10.89\k\Omega$
		\item $V_{\alpha ab}$ is now a simple voltage divider $\displaystyle{V_{\alpha ab}=6\frac{47\k\Omega}{47\k\Omega+10.89\k\Omega}}$
		\item $V_{\alpha ab}=4.871\V$
		      \\
		      \\
		      \\
		      \\
		      \\
	\end{enumerate}
	\begin{enumerate}
		\item We add the resistors in series in the top-left loop $33\k\Omega+6.8\k\Omega=39.8\k\Omega$
		\item Next we calculate the bottom-left loop parallel resistance $15\k\Omega||47\k\Omega=11.37\k\Omega$
		\item Now we have two voltage dividers, be careful to substract the $11.37\k\Omega$ or $2.2\k\Omega$ resistor voltages due to the passive sign convention.
		\item $\displaystyle{V_{\beta ab}=6\V\left(\frac{1\k\Omega}{1\k\Omega+2.2\k\Omega}-\frac{11.37\k\Omega}{11.37\k\Omega+39.8\k\Omega}\right)}$
		\item Note, we could have used the $2.2\k\Omega$ and $39.8\k\Omega$ route if we wanted.
		\item $V_{\beta ab}=541.8\m\V$
	\end{enumerate}
\end{multicols}

\begin{multicols}{2}
	\begin{center}
		\begin{circuitikz}
			% NW
			\draw
			(3,3) to[R, l_=$10.89\k\Omega$]
			(0,3);
			% NE
			\draw
			(3,3) to [short, -o]
			(4,3) ++ (16pt,0) node[left] {a};
			% SW
			\draw
			(0,3) to[V, l=$6\V$]
			(0,0) --
			(3,0) to[R, l_=$47\k\Omega$]
			(3,3);
			\draw
			(2,0) to (2,-0.25) node[tlground] {};
			\draw
			(3,0) to [short, -o]
			(4,0) ++ (16pt,0) node[left] {b};
		\end{circuitikz}
		\captionof{figure}{Open-circuit with $V_\alpha$ (simplified)}
	\end{center}
	\begin{center}
		\begin{circuitikz}
			% NW
			\draw
			(0,0) --
			(0,3) to[R, l_=$39.8\k\Omega$]
			(3,3) to[V, l=$6\V$]
			(3,0) to[R, l_=$11.37\k\Omega$]
			(0,0);
			% NE
			\draw
			(3,3) to[R, l_=$2.2\k\Omega$]
			(6,3) --
			(6,0) to[R, l_=$1\k\Omega$]
			(3,0);
			\draw
			(6,3) to [short, -o]
			(6,3.5) ++ (-16pt,0) node[right] {a};
			\draw
			(0,0) to (0,-0.25) node[tlground] {};
			\draw
			(0,3) to [short, -o]
			(0,3.5) ++ (-16pt,0) node[right] {b};
		\end{circuitikz}
		\captionof{figure}{Open-circuit with $V_\beta$ (simplified)}
	\end{center}
\end{multicols}

$$V_{ab}=V_{\alpha ab}+V_{\beta ab}=5.41\V$$

\clearpage

\subsection{Simulated}

\clearpage

\section{Short-circuit current}

\subsection{Nodal analysis}

\begin{figure}[ht]
	\begin{center}
		\begin{circuitikz}
			% NW
			\draw
			(0,3) to[R, l_=$33\k\Omega$]
			(0,6) to[R, l=$6.8\k\Omega$]
			(4,6) to[V, l=$6\V$]
			(4,3) to[R, l_=$15\k\Omega$]
			(0,3);

			% NE
			\draw
			(4,6) to[R, l=$2.2\k\Omega$]
			(8,6) --
			(8,3) to[R, l_=$1\k\Omega$]
			(4,3);

			\draw
			(8,3) to [short, -*]
			(8,2.5) ++ (16pt,0) node[left] {a};

			% SW
			\draw
			(0,3) to[V, l=$6\V$]
			(0,0) --
			(4,0) to[R, l_=$47\k\Omega$]
			(4,3);

			\draw
			(2,0) to (2,-0.25) node[tlground] {};

			% SE
			\draw
			(4,0) to [R, l=$560\Omega$]
			(8,0) --
			(8,3);

			\draw
			(8,0) to [short, -*]
			(8,0.5) ++ (16pt,0) node[left] {b};

			% nodal
			\draw (4,6) to [short, -*] (4,6);
			\draw (4,6.5) ++ (16pt,0) node[left] {$V_1$};
			\draw (4,3) to [short, -*] (4,3);
			\draw (4,3.5) ++ (16pt,0) node {$V_2$};
			\draw (0,3) to [short, -*] (0,3);
			\draw (-1,3) ++ (16pt,0) node[left] {$V_3$};
			\draw (8,6) to [short, -*] (8,6);
			\draw (7.5,6) ++ (16pt,0) node[right] {$V_4$};
		\end{circuitikz}
		\caption{Short-circuit with nodes added}
	\end{center}
\end{figure}

\begin{enumerate}
	\item The number of nodal equations required equals: \\ (essential nodes - ground node - voltage source) $= 5-1-2=2$
	\item Determine our node locations, marked on the diagram $V_{(1-4)}$ we can easily deduced $V_3=6\V$, and $V_1=V_2+6\V$ making $V_1$ and $V_2$ a supernode.
	\item \begin{align*}
		      0 & =\frac{(V_2+6\V)-6\V}{6.8\k\Omega+33\k\Omega}
		      +\frac{(V_2+6\V)-V_4}{2.2\k\Omega}
		      +\frac{V_2-V_4}{1\k\Omega}
		      +\frac{V_2}{47\k\Omega}
		      +\frac{V_2-6\V}{15\k\Omega}                       \\
		      0 & =\frac{V_4-(V_2+6\V)}{2.2\k\Omega}
		      +\frac{V_4-V_2}{1\k\Omega}
		      +\frac{V_4}{560\Omega}
	      \end{align*}
	\item \begin{align*}
		      \frac{6\V}{15\k\Omega}-\frac{6\V}{2.2\k\Omega} & =V_2\left(\frac{1}{39.8\k\Omega}
		      +\frac{1}{2.2\k\Omega}
		      +\frac{1}{1\k\Omega}
		      +\frac{1}{47\k\Omega}
		      +\frac{1}{15\k\Omega}\right)
		      +V_4\left(-\frac{1}{2.2\k\Omega}
		      -\frac{1}{1\k\Omega}\right)
		      \\
		      \frac{6\V}{2.2\k\Omega}                        & =
		      V_2\left(-\frac{1}{2.2\k\Omega}-\frac{1}{1\k\Omega}\right)
		      +V_4\left(\frac{1}{2.2\k\Omega}
		      +\frac{1}{1\k\Omega}
		      +\frac{1}{560\Omega}\right)
	      \end{align*}
	\item \[
		      \begin{bmatrix}
			      1.568\m\Omega^{-1}  & -1.455\m\Omega^{-1} \\
			      -1.455\m\Omega^{-1} & 3.240\m\Omega^{-1}  \\
		      \end{bmatrix}
		      \begin{bmatrix}
			      V_2 \\
			      V_4 \\
		      \end{bmatrix}
		      =
		      \begin{bmatrix}
			      -2.327\m\A \\
			      2.727\m\A  \\
		      \end{bmatrix}
	      \]
	\item $V_2=-1.205\V$, $V_4=300.4\m\V$, $V_1=4.795\V$
	\item $\displaystyle{I_{ab}=\frac{V_1-V_4}{2.2\k\Omega}+\frac{V_2-V_4}{1\k\Omega}=\frac{(4.795-0.3004)\V}{2.2\k\Omega}+\frac{(-1.205-0.3004)\V}{1\k\Omega}}=537\u\A$
\end{enumerate}

\clearpage

\subsection{Mesh analysis}

\begin{figure}[ht]
	\begin{center}
		\begin{circuitikz}
			% NW
			\draw
			(0,3) to[R, l_=$33\k\Omega$]
			(0,6) to[R, l=$6.8\k\Omega$]
			(4,6) to[V, l=$6\V$]
			(4,3) to[R, l_=$15\k\Omega$]
			(0,3);

			% NE
			\draw
			(4,6) to[R, l=$2.2\k\Omega$]
			(8,6) to[R, l=$1\k\Omega$]
			(8,3) --
			(4,3);

			\draw
			(8,6) to[short, -o]
			(9,6) ++(16pt,0) node[left] {a};

			% SW
			\draw
			(0,3) to[V, l=$6V$]
			(0,0) --
			(4,0) to[R, l_=$47\k\Omega$]
			(4,3);

			\draw
			(2,0) to (2,-0.25) node[tlground] {};

			% SE
			\draw
			(4,0) to [R, l=$560\Omega$]
			(8,0);

			\draw
			(8,0) to [short, -o]
			(9,0) ++ (16pt,0) node[left] {b};

			% mesh
			\draw[->] (1.6,5) arc(120:-120:6mm) node[midway, left, font=\footnotesize] {$i_1(t)$};
			\draw[->] (1.6,2) arc(120:-120:6mm) node[midway, left, font=\footnotesize] {$i_2(t)$};
			\draw[->] (5.6,5) arc(120:-120:6mm) node[midway, left, font=\footnotesize] {$i_3(t)$};
		\end{circuitikz}
		\caption{Short-circuit with mesh currents added}
	\end{center}
\end{figure}

\begin{enumerate}
	\item Determine our mesh currents labeled $i_{1-3}$
	\item \begin{align*}
		      0 & =33\k\Omega(i_1)+6.8\k\Omega(i_1)+6\V+15\k\Omega(i_1-i_2) \\
		      0 & =-6\V+15\k\Omega(i_2-i_1)+47\k\Omega(i_2)                 \\
		      0 & =-6\V+2.2\k\Omega(i_3)+1\k\Omega(i_3)
	      \end{align*}
	\item \begin{align*}
		      -6\V & = 54.8\k\Omega(i_1)-15\k\Omega(i_2) \\
		      6\V  & = -15\k\Omega(i_1)+62\k\Omega(i_2)  \\
		      6\V  & = 3.2\k\Omega(i_3)
	      \end{align*}
	\item \[
		      \begin{bmatrix}
			      54.8\k\Omega & -15\k\Omega & 0           \\
			      -15\k\Omega  & 62\k\Omega  & 0           \\
			      0            & 0           & 3.2\k\Omega \\
		      \end{bmatrix}
		      \begin{bmatrix}
			      i_1 \\
			      i_2 \\
			      i_3 \\
		      \end{bmatrix}
		      =
		      \begin{bmatrix}
			      -6\V \\
			      6\V  \\
			      6\V  \\
		      \end{bmatrix}
	      \]
	\item $i_1=-88.9\u\A$, $i_2=75.3\u\A$, $i_3=1.88\m\A$
	\item There is no current through the $560\Omega$ resistor so we know $V_{ab}=1\k\Omega(i_3)+47\k\Omega(i_2)$
	\item $V_{ab}=1\k\Omega(1.88\m\A)+47\k\Omega(75.3\u\A)=5.41\V$
\end{enumerate}

\clearpage

\subsection{Superposition}

\begin{multicols}{2}
	\begin{center}
		\begin{circuitikz}
			% NW
			\draw
			(0,3) to[R, l_=$33\k\Omega$]
			(0,6) to[R, l=$6.8\k\Omega$]
			(3,6) --
			(3,3) to[R, l_=$15\k\Omega$]
			(0,3);
			% NE
			\draw
			(3,6) to[R, l=$2.2\k\Omega$]
			(6,6) --
			(6,3) to[R, l_=$1\k\Omega$]
			(3,3);
			\draw
			(6,3) to [short, -o]
			(6,2) ++ (-16pt,0) node[right] {a};
			% SW
			\draw
			(0,3) to[V, l=$6\V$]
			(0,0) --
			(3,0) to[R, l_=$47\k\Omega$]
			(3,3);
			\draw
			(2,0) to (2,-0.25) node[tlground] {};
			\draw
			(3,0) to [short, -o]
			(4,0) ++ (16pt,0) node[left] {b};
		\end{circuitikz}
		\captionof{figure}{Short-circuit with $V_\alpha$ isolated}
	\end{center}
	\begin{center}
		\begin{circuitikz}
			% NW
			\draw
			(0,3) to[R, l_=$33\k\Omega$]
			(0,6) to[R, l=$6.8\k\Omega$]
			(3,6) to[V, l=$6\V$]
			(3,3) to[R, l_=$15\k\Omega$]
			(0,3);
			% NE
			\draw
			(3,6) to[R, l=$2.2\k\Omega$]
			(6,6) --
			(6,3) to[R, l_=$1\k\Omega$]
			(3,3);
			\draw
			(6,3) to [short, -o]
			(6,2) ++ (-16pt,0) node[right] {a};
			% SW
			\draw
			(0,3) --
			(0,0) --
			(3,0) to[R, l_=$47\k\Omega$]
			(3,3);
			\draw
			(2,0) to (2,-0.25) node[tlground] {};
			\draw
			(3,0) to [short, -o]
			(4,0) ++ (16pt,0) node[left] {b};
		\end{circuitikz}
		\captionof{figure}{Short-circuit with $V_\beta$ isolated}
	\end{center}
\end{multicols}

\begin{multicols}{2}
	\begin{enumerate}
		\item The top-right loop has no current through it, so it can be elided.
		\item The top-left loop can now be simplified: $R_{eq}=(33\k\Omega+6.8\k\Omega)||15\k\Omega=10.89\k\Omega$
		\item $V_{\alpha ab}$ is now a simple voltage divider $\displaystyle{V_{\alpha ab}=6\frac{47\k\Omega}{47\k\Omega+10.89\k\Omega}}$
		\item $V_{\alpha ab}=4.871\V$
		      \\
		      \\
		      \\
		      \\
		      \\
	\end{enumerate}
	\begin{enumerate}
		\item We add the resistors in series in the top-left loop $33\k\Omega+6.8\k\Omega=39.8\k\Omega$
		\item Next we calculate the bottom-left loop parallel resistance $15\k\Omega||47\k\Omega=11.37\k\Omega$
		\item Now we have two voltage dividers, be careful to substract the $11.37\k\Omega$ or $2.2\k\Omega$ resistor voltages due to the passive sign convention.
		\item $\displaystyle{V_{\beta ab}=6\V\left(\frac{1\k\Omega}{1\k\Omega+2.2\k\Omega}-\frac{11.37\k\Omega}{11.37\k\Omega+39.8\k\Omega}\right)}$
		\item Note, we could have used the $2.2\k\Omega$ and $39.8\k\Omega$ route if we wanted.
		\item $V_{\beta ab}=541.8\m\V$
	\end{enumerate}
\end{multicols}

\begin{multicols}{2}
	\begin{center}
		\begin{circuitikz}
			% NW
			\draw
			(3,3) to[R, l_=$10.89\k\Omega$]
			(0,3);
			% NE
			\draw
			(3,3) to [short, -o]
			(4,3) ++ (16pt,0) node[left] {a};
			% SW
			\draw
			(0,3) to[V, l=$6\V$]
			(0,0) --
			(3,0) to[R, l_=$47\k\Omega$]
			(3,3);
			\draw
			(2,0) to (2,-0.25) node[tlground] {};
			\draw
			(3,0) to [short, -o]
			(4,0) ++ (16pt,0) node[left] {b};
		\end{circuitikz}
		\captionof{figure}{Short-circuit with $V_\alpha$ (simplified)}
	\end{center}
	\begin{center}
		\begin{circuitikz}
			% NW
			\draw
			(0,0) --
			(0,3) to[R, l_=$39.8\k\Omega$]
			(3,3) to[V, l=$6\V$]
			(3,0) to[R, l_=$11.37\k\Omega$]
			(0,0);
			% NE
			\draw
			(3,3) to[R, l_=$2.2\k\Omega$]
			(6,3) --
			(6,0) to[R, l_=$1\k\Omega$]
			(3,0);
			\draw
			(6,3) to [short, -o]
			(6,3.5) ++ (-16pt,0) node[right] {a};
			\draw
			(0,0) to (0,-0.25) node[tlground] {};
			\draw
			(0,3) to [short, -o]
			(0,3.5) ++ (-16pt,0) node[right] {b};
		\end{circuitikz}
		\captionof{figure}{Short-circuit with $V_\beta$ (simplified)}
	\end{center}
\end{multicols}

$$V_{ab}=V_{\alpha ab}+V_{\beta ab}=5.41\V$$

\clearpage

\subsection{Simulated}

\section{Comparison}
\section{Th\'evenin Equivalent Circuit}

\end{document}
