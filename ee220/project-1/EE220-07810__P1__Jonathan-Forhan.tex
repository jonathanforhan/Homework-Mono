% Jonathan Forhan
% EE-220 Project 1

\documentclass[12pt]{article}

\title{EE-220 Project 1}
\author{Jonathan Forhan\\ EE220-07810\\ Southern New Hampshire University\\ Spring 2024}
\date{ }

\usepackage{geometry}
\usepackage{tocloft}

% toc dots
\renewcommand{\cftsecleader}{\cftdotfill{\cftdotsep}}

\begin{document}

\maketitle
\tableofcontents
\thispagestyle{empty}
\clearpage

\section{Objective}

The objective of this project is to build a DC motor out of basic materials.

\section{Materials}

\begin{enumerate}
	\item[$\bullet$] AA battery
	\item[$\bullet$] Battery mount with alligator clips
	\item[$\bullet$] Breadboard
	\item[$\bullet$] Insulated wire
	\item[$\bullet$] Wire cutters
	\item[$\bullet$] Sandpaper
	\item[$\bullet$] Neodymium magnet
\end{enumerate}

\section{Procedure}

\begin{enumerate}
	\item Coil wire around AA battery at least 10 times. Leave 2in on each ends.
	\item Remove wire coil from battery and wrap ends around itself two times.
	\item Remove insulation from ends of coil. Be careful not to damage the wire.
	\item Cut two pieces of wire about 8in, then remove 1in of insulation from each end.
	\item Bend one end of each 8in wire and form an 'S' shape.
	\item Secure the other end of the wires into breadboard, about 2in apart.
	\item Center the coil on the 'S' shaped end of the wires. Center and balance the coil.
	\item Place magnet underneath the coil.
	\item Attach the alligator clips with the end of the 8in wires, passing a current through the coil.
\end{enumerate}

\section{Results}

When the circuit is closed then the wire rotates.

\section{Analysis}

\section{Discussion}

\section{Conclusion}

\end{document}
