% Jonathan Forhan
% EE-220 Project 1

\documentclass[12pt]{article}

\title{EE-220 Project 1}
\author{Jonathan Forhan\\ EE220-07810\\ Southern New Hampshire University\\ Spring 2024}
\date{ }

\usepackage{geometry}
\usepackage{tocloft}
\usepackage[none]{hyphenat}

% toc dots
\renewcommand{\cftsecleader}{\cftdotfill{\cftdotsep}}

\begin{document}

\maketitle
\tableofcontents
\thispagestyle{empty}
\clearpage

\section{Objective}

This project aims to construct a functional DC motor using readily available materials,
demonstrating the conversion of electrical energy into mechanical motion.

\section{Materials}

\begin{enumerate}
	\item[$\bullet$] AA battery
	\item[$\bullet$] Battery mount with alligator clips
	\item[$\bullet$] Breadboard
	\item[$\bullet$] Insulated wire
	\item[$\bullet$] Wire cutters
	\item[$\bullet$] Sandpaper
	\item[$\bullet$] Neodymium magnet
\end{enumerate}

\section{Procedure}

\begin{enumerate}
	\item Coil wire around AA battery at least 10 times. Leave 2in on each ends.
	\item Remove wire coil from battery and wrap ends around itself two times.
	\item Remove insulation from ends of coil. Be careful not to damage the wire.
	\item Cut two pieces of wire about 8in, then remove 1in of insulation from each end.
	\item Bend one end of each 8in wire and form an 'S' shape.
	\item Secure the other end of the wires into breadboard, about 2in apart.
	\item Center the coil on the 'S' shaped end of the wires. Center and balance the coil.
	\item Place magnet underneath the coil.
	\item Attach the alligator clips with the end of the 8in wires, passing a current through the coil.
\end{enumerate}

\section{Results}

Closing the circuit initiated rotational motion of the wire coil, confirming the successful
operation of the DC motor. Overall, the results affirm the project's objectives,
demonstrating the conversion of electrical energy into mechanical motion using basic
materials.

\section{Analysis}

When current flows through the coil, it generates a magnetic
field according to Ampère's law. Simultaneously, the neodymium magnet establishes
a static magnetic field. The interaction between these magnetic fields creates a force,
known as the Lorentz force, which acts perpendicular to the current and the magnetic
field, causing the coil to rotate. The number of coil turns and the strength of the
magnetic field influence the torque exerted on the coil. This experiment demonstrates
the conversion of electrical energy into mechanical motion, illustrating the practical
application of electromagnetism in DC motor construction.

\section{Discussion and Conclusion}

This project successfully demonstrated the conversion of electrical energy into mechanical
motion through the construction of a basic DC motor. By following a straightforward
procedure, I witnessed the interplay between current flow and magnetic fields,
resulting in the rotation of the wire coil. \\ \\
The observed rotation confirmed the operation of the motor, driven by the torque
generated by the interaction of magnetic fields. Key factors influencing performance
include the number of coil turns and the strength of the magnetic field. \\ \\
In essence, this project provided practical insight into electromagnetism principles
and their application in motor construction. It lays the groundwork for further
exploration in electromagnetics.

\end{document}
