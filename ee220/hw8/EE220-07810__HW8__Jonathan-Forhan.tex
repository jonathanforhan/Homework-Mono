\documentclass[14pt]{extarticle}

\usepackage[margin=0.5in]{geometry}
\usepackage{amsmath}
\usepackage[none]{hyphenat}
\usepackage{pgfplots}

\pgfplotsset{compat=newest}

\title{EE-220 Homework 8}
\author{Jonathan Forhan}
\date{ }

\renewcommand{\thesubsection}{\thesection-\alph{subsection}}

\begin{document}

\maketitle

\boldmath
\section{Consider a point-charge $Q_1(0 ,-4 ,0)$ and a point-\\
  charge $Q_2(0, 4, 0)$, each $2\mathrm{nC}$.}
\unboldmath

\begin{center}
	\begin{tikzpicture}
		\begin{axis}[
				hide axis,
				view={120}{15},
				xmax=4,
				ymax=4,
				zmax=4,
				xmin=-4,
				ymin=-4,
				zmin=-4,
				xlabel=$x$,
				ylabel=$y$,
				zlabel=$z$,
			]
			\addplot3[color=blue,mark=*,mark size=1] table[row sep=crcr] { 0 0 -10 \\ 0 0 10 \\ };
			\addplot3[color=green,mark=*,mark size=1] table[row sep=crcr] { 0 -10 0 \\ 0 10 0 \\ };
			\addplot3[color=red,mark=*,mark size=1] table[row sep=crcr] { -10 0 0 \\ 10 0 0 \\ };
			\addplot3[mark=*,black,point meta=explicit symbolic,nodes near coords] coordinates {(0,-4,0)[$Q_1(0,-4,0)$]};
			\addplot3[mark=*,black,point meta=explicit symbolic,nodes near coords] coordinates {(0,4,0)[$Q_2(0,4,0)$]};
			\addplot3[mark=*,black,point meta=explicit symbolic,nodes near coords] coordinates {(0,0,0)[$$]};
			\addplot3[mark=*,black,point meta=explicit symbolic,nodes near coords] coordinates {(0,0,4)[$P(0,0,4)$]};
		\end{axis}
	\end{tikzpicture}
\end{center}

\boldmath
\subsection{Calculate the potential at the origin.}
\unboldmath

$$V=\frac{Q}{4\pi\epsilon_0 r}$$

$$V=\frac{2\mathrm{nC}}{4\pi\epsilon_0(4)}+\frac{2\mathrm{nC}}{4\pi\epsilon_0(4)}=\frac{1\mathrm{nC}}{4\pi\epsilon_0}=
	\frac{10^{-9}}{4\pi\frac{10^{-9}}{36\pi}}=9[\mathrm{V}]$$

\boldmath
\subsection{Calculate the potential at the point P(0,0,4).}
\unboldmath

$$V=\frac{2\mathrm{nC}}{4\pi\epsilon_0(\sqrt{32})}+\frac{2\mathrm{nC}}{4\pi\epsilon_0(\sqrt{32})}=\frac{4\mathrm{nC}}{16\sqrt{2}\pi\epsilon_0}=
	\frac{1\mathrm{nC}}{4\sqrt{2}\pi\epsilon_0}=\frac{10^{-9}}{4\sqrt{2}\pi\frac{10^{-9}}{36\pi}}$$

$$V=\frac{9\sqrt{2}}{2}=6.36[\mathrm{V}]$$

\boldmath
\subsection{What is the voltage of the origin with respect to the point $P,V_{OP}$?}
\unboldmath

$$V_{OP}=9-6.36=2.64[\mathrm{V}]$$

\boldmath
\subsection{Calculate the amount of energy required to move a $10\mathrm{nC}$ charge from point P to the origin.}
\unboldmath

$$W=QV$$
$$W_{OP}=10nC\cdot 2.64=26.4[\mathrm{nJ}]$$

\boldmath
\section{The length of each piece of jumper wire in the lab
  is about 5.0cm. Assuming AWG-20 (wire diameter
  0.812mm) copper wire with $\sigma=5.8\times10^7\left[\mathrm{S/m}\right]$}
\unboldmath

\boldmath
\subsection{Determine the resistance of each jumper wire.}
\unboldmath

$$\rho = \frac{1}{\sigma}$$

$$\rho = \frac{1}{5.8\times10^7}=5.8\times10^{-7}[\Omega\mathrm{m}]$$

\boldmath
\subsection{Determine the power dissipated in the wire for 10 mA of current.}
\unboldmath

$$P=I^2\times R$$
$$R=\rho L A$$
$$P=(10\times10^{-3})^2\times\left(5.8 \times 10^{-7}\right)(5^{-2})(\pi \times (0.406^{-3})^2)$$
$$P=1.627[\mathrm{nW}]$$

\clearpage

\boldmath
\section{The force measured between a pair of point charges
  separated by a dielectric material is 20 nN. If the
  dielectric material is removed without changing
  the position of the point charges, the force increases
  to 100 nN.}
\unboldmath

\boldmath
\subsection{Calculate the relative permittivity $\epsilon_r$ of the dielectric.}
\unboldmath

$$F=\frac{1}{\epsilon}\cdot\frac{Q\cdot q}{4\pi d^2}$$

$$F=\frac{1}{\epsilon_0\epsilon_r}\cdot\frac{Q\cdot q}{4\pi d^2}$$
$$F_0=\frac{1}{\epsilon_0}\cdot\frac{Q\cdot q}{4\pi d^2}$$

\begin{center}
	$\frac{Q\cdot q}{4\pi d^2}$ cancels.
\end{center}

$$F=\frac15 F_0\rightarrow\frac{1}{\epsilon_0\epsilon_r}=\frac{1}{5\epsilon_0}\rightarrow\epsilon_r = 5$$

$$\epsilon_r=5$$

\end{document}
